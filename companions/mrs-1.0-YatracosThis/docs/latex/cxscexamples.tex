\-To learn how to use the \-C-\/\-X\-S\-C class library and the \-G\-S\-L you can read the page \hyperlink{cxscexamples}{\-Basic \-C-\/\-X\-S\-C \-Examples} and follow the examples described there. \-For a proper introduction to \-C-\/\-X\-S\-C and the accompanying \-Toolbox with more sophisticated applications see the \href{http://www.math.uni-wuppertal.de/wrswt/xsc/cxsc/apidoc/html/index.html}{\tt \-C-\/\-X\-S\-C \-Documentation}.

\-On this page we will try to show you how to use the \-C-\/\-X\-S\-C class library with some short and simple examples.


\begin{DoxyItemize}
\item \hyperlink{cxscexamples_cxscexamples_sec_ex1}{\-Example 1 -\/ \-Intervals in \-C-\/\-X\-S\-C}
\item \hyperlink{cxscexamples_cxscexamples_sec_ex2}{\-Example 2 -\/ \-Multi-\/precision \-Intervals in \-C-\/\-X\-S\-C}
\item \hyperlink{cxscexamples_cxscexamples_sec_ex3}{\-Example 3 -\/ \-Mean in \-G\-S\-L and \-Dot \-Precision \-Accumulators in \-C-\/\-X\-S\-C}
\item \hyperlink{cxscexamples_cxscexamples_sec_ex4}{\-Example 4 -\/ \-Range \-Enclosure with \-Automatic \-Differentiation}
\item \hyperlink{cxscexamples_cxscexamples_sec_ex5}{\-Example 5 -\/ \-Global \-Optimisation for \-Maximum \-Likelihood}
\end{DoxyItemize}\hypertarget{cxscexamples_cxscexamples_sec_ex1}{}\subsection{\-Example 1 -\/ Intervals in C-\/\-X\-S\-C}\label{cxscexamples_cxscexamples_sec_ex1}
\-In the following simple program we use \-C-\/\-X\-S\-C intervals to do basic arithmetic operations and output the results\-:




\begin{DoxyCodeInclude}
#include "interval.hpp"  // include interval arithmetic in C-XSC
#include <iostream>      // include standard Input Output STREAM 
using namespace cxsc;    
using namespace std;

int main()
{
  interval a, b;            // Standard intervals     
  a = 1.0;                  // a   = [1.0,1.0]       
  "[1, 2]" >> b;          // string to interval conversion b   = [1.0,2.0]     
         
  cout << "a - a = " << a-a << endl;
  cout << "b - b = " << b-b << endl;
}

/* --------------------------- Output ------------------------------
$ ./example 
a - a = [ -0.000000,  0.000000]
b - b = [ -1.000000,  1.000000]
------------------------------------------------------------------*/

\end{DoxyCodeInclude}




\-Let's start examining the code line by line. \-The first line\-:


\begin{DoxyCodeInclude}
#include "interval.hpp"  // include interval arithmetic in C-XSC

\end{DoxyCodeInclude}


includes the basic interval class of \-C-\/\-X\-S\-C in the program. \-The second line\-:


\begin{DoxyCodeInclude}
#include <iostream>      // include standard Input Output STREAM 

\end{DoxyCodeInclude}


includes the standard iostream library for basic input and output operations. \-The next two lines inform the compiler about \-C-\/\-X\-S\-C's namespace cxsc and the standard library namespace std. \-The namespace cxsc is where all of \-C-\/\-X\-S\-C's classes and methods are stored. \-This allows us to use \-C-\/\-X\-S\-C classes without having to fully qualify their identifiers.


\begin{DoxyCodeInclude}
using namespace cxsc;    
using namespace std;

\end{DoxyCodeInclude}


\-Next we declare two interval variables and assign adequate values in the following lines.


\begin{DoxyCodeInclude}
  interval a, b;            // Standard intervals     
  a = 1.0;                  // a   = [1.0,1.0]       
  "[1, 2]" >> b;          // string to interval conversion b   = [1.0,2.0]     
         

\end{DoxyCodeInclude}


\-Finally, we print out the result for our desired subtractions.


\begin{DoxyCodeInclude}
  cout << "a - a = " << a-a << endl;
  cout << "b - b = " << b-b << endl;

\end{DoxyCodeInclude}


\-To compile the program we edit the \-Makefile in the examples directory. \-First we set the '\-P\-R\-O\-G\-R\-A\-M=example' and the \-P\-R\-E\-F\-I\-X to the appropriate directory that contains the \-C-\/\-X\-S\-C includes and lib directories. \-Then we type 'make all' in a \-Unix system to compile the program.\hypertarget{cxscexamples_cxscexamples_sec_ex2}{}\subsection{\-Example 2 -\/ Multi-\/precision Intervals in C-\/\-X\-S\-C}\label{cxscexamples_cxscexamples_sec_ex2}



\begin{DoxyCodeInclude}
#include "l_interval.hpp"  // interval staggered arithmetic in C-XSC
#include <iostream>
using namespace cxsc;
using namespace std;

int main() 
{
  l_interval a, b;         // Multiple-precision intervals in C-XSC
  stagprec = 2;            // global integer variable      
  cout << SetDotPrecision(16*stagprec, 16*stagprec-3) << RndNext;
  // I/O for variables of type l_interval is done using
  // the long accumulator (i.e. a dotprecision variable)   

  a = 1.0;                  // a   = [1.0,1.0]       
  "[1, 2]" >> b;          // string to interval conversion b   = [1.0,2.0]     
         
  cout << "a - a = " << a-a << endl;
  cout << "b - b = " << b-b << endl;
  cout << "a/b = " << a/b << endl;  
}

/* --------------------------- Output ------------------------------
$ ./lexample 
a - a = [ 0.00000000000000000000000000000, 0.00000000000000000000000000000]
b - b = [-1.00000000000000000000000000000, 1.00000000000000000000000000000]
a/b = [ 0.50000000000000000000000000000, 1.00000000000000000000000000000]
------------------------------------------------------------------*/

\end{DoxyCodeInclude}
\hypertarget{cxscexamples_cxscexamples_sec_ex3}{}\subsection{\-Example 3 -\/ Mean in G\-S\-L and Dot Precision Accumulators in C-\/\-X\-S\-C}\label{cxscexamples_cxscexamples_sec_ex3}
\hypertarget{cxscexamples_cxscexamples_sec_ex4}{}\subsection{\-Example 4 -\/ Range Enclosure with Automatic Differentiation}\label{cxscexamples_cxscexamples_sec_ex4}
\hypertarget{cxscexamples_cxscexamples_sec_ex5}{}\subsection{\-Example 5 -\/ Global Optimisation for Maximum Likelihood}\label{cxscexamples_cxscexamples_sec_ex5}
